\chapter{INTRODUCTION}
% (20% of Proposal Length)
\pagenumbering{arabic}

% Introduction: (20\% of Report Length)


\section{Introduction}
In today’s digital era, the rapid spread of fake news poses a serious threat to public trust and societal well-being. TrueLens: Fake News Detection System offers a practical solution by using advanced machine learning techniques to identify and flag misleading information. Powered by Python and TensorFlow, TrueLens integrates seamlessly with Django to provide a user-friendly interface where individuals can verify the authenticity of news articles. With efficient data management via PostgreSQL and robust version control through GitHub, TrueLens stands as a crucial tool in the fight against misinformation, promoting informed decision-making and enhancing the integrity of online content.


\section{Problem Statement}

In the digital age, the rampant spread of fake news undermines public trust and distorts reality. The volume and sophistication of misinformation on digital platforms make it difficult for individuals to discern fact from fiction. Traditional verification methods are often inadequate, leading to widespread misinformation with significant societal and political impacts. There is a lack of effective, user-friendly tools for quick and accurate news verification, which exacerbates the issue.

Additionally, there is a significant educational gap, with many users unaware of how to critically evaluate fake news. Current solutions either lack real-time detection capabilities or are too complex for general use. This highlights the need for a robust system that not only detects fake news using advanced machine learning but also educates users about misinformation. TrueLens aims to fill this gap by providing an accessible, efficient, and educational platform that empowers users to combat false information and make informed decisions.


\section{Objectives}
\begin{itemize}
    \item  To empower users with a user-friendly tool for swiftly and accurately verifying news authenticity, thereby combating the spread of misinformation effectively.
\end{itemize}
\section{Scope}
% Scope and limitation

\begin{itemize}
    \item Implement advanced machine learning algorithms to accurately detect and classify fake news articles based on linguistic and statistical analysis.
    \item Develop a user-friendly web interface that allows users to easily submit news articles for verification and receive clear, understandable results regarding their authenticity.
    \item Provide educational resources within the platform to enhance users' understanding of fake news, promoting critical thinking and media literacy skills among the general public.
    
\end{itemize}
\section{Report Organisation}
The material in this project report is organised into seven chapters. After this introductory chapter introduces the problem topic this research tries to address, chapter 2 contains the literature review of vital and relevant publications, pointing toward a notable research gap. Chapter 3 describes the methodology for the implementation of this project. Chapter 4 provides an overview of what has been accomplished. Chapter 5 contains some crucial discussions on the used model and methods. Chapter 6 mentions pathways for future research direction for the same problem or in the same domain. Chapter 7 concludes the project shortly, mentioning the accomplishment and comparing it with the main objectives.