\chapter{BACKGROUND AND LITERATURE REVIEW}

% (20\% of Report Length)

% a. Must be paraphrased without plagiarizing

% b. Must include the base papers\cite{Adhikari2020Dec}, and support the rationale of the project

% c. Must highlight the strengths and shortcomings of the works performed by other authors

\section{Background Study}

In today's digital age, the rapid spread of fake news on social media and other online platforms has become a critical issue, influencing public opinion and eroding trust in genuine news sources. Traditional fact-checking methods are often too slow to combat the swift spread of misinformation. Machine learning offers a powerful solution by analyzing text for patterns that indicate falsehoods, but many existing tools are either complex or lack real-time capabilities. TrueLens: Fake News Detection System addresses these challenges by providing a user-friendly platform that uses advanced machine learning to quickly verify the authenticity of news articles, thereby helping users navigate the information landscape more effectively and promoting informed decision-making.

\section{Limitation}
\begin{itemize}
    \item The system may struggle with detecting fake news if its training data is biased or lacks diversity, limiting its effectiveness in identifying new or less common misinformation.
    \item The system can misinterpret nuanced language or context, potentially leading to inaccuracies in detecting sarcasm, subtle biases, or contextually misleading information.
\end{itemize}
\section{Literature Review}
The exponential growth of digital media and social networks has led to an unprecedented increase in the spread of fake news, significantly affecting public opinion and trust in credible news sources. It highlight how the rapid dissemination of fake news, particularly during the 2016 U.S. Presidential Election, influenced voter behavior and public perceptions, emphasizing the need for robust mechanisms to detect and mitigate the spread of misinformation. Their study sheds light on the economic and social repercussions of fake news, which necessitates the development of effective detection systems to maintain information integrity in the digital era.\cite{allcott2017social}\\\\
Machine learning has emerged as a crucial tool in the detection of fake news, offering the ability to analyze large datasets and identify patterns that human analysis might miss. Conroy, Rubin, and Chen (2015) classify fake news detection methodologies into three main categories: linguistic, network-based, and hybrid approaches. Linguistic approaches focus on analyzing the language and writing style of news articles to identify deceptive content, while network-based methods examine the dissemination patterns and relationships between news articles and their sources. Hybrid approaches combine linguistic and network-based techniques to enhance detection accuracy, leveraging the strengths of both methodologies to combat misinformation effectively.\cite{conroy2015automatic}\\\\
Despite these advancements, challenges remain in developing effective fake news detection systems. Shu, Wang, and Liu (2017) discuss the difficulties associated with the evolving nature of fake news, which requires continuous updates to detection models to keep up with new types of misinformation. They also emphasize the need for diverse and high-quality datasets to train and test detection algorithms effectively. Zhou and Zafarani (2018) highlight the scarcity of comprehensive labeled datasets for fake news detection, which limits the ability to develop and validate robust models. The TrueLens project aims to address these challenges by integrating advanced machine learning techniques with a user-friendly web interface, providing an accessible and scalable solution for fake news detection.\cite{zhou2018survey}\\\\
Deep learning has significantly advanced fake news detection models. Ruchansky, Seo, and Liu (2017) present the ”CSI” model, a hybrid deep learning framework integrating content analysis, social context, and user behavior to detect fake news more accurately. Their approach combines diverse data sources and analytical techniques, improving detection compared to traditional methods by capturing the dynamic spread of misinformation and its social influences.\cite{ruchansky2017csi}